%\documentclass{article}
%\makeindex
%\usepackage{makeidx}

%\begin{document}

\chapter{BProc Messages}

This chapter describes the messages that BProc uses to communicate
between the kernel layers and the master and slave daemons.  BProc
messages fixed size as defined by \texttt{struct bproc\_request\_t} in
\texttt{kernel/bproc.h}.  This structure and the message types are not
intended to be a stable interface.  This is strictly a BProc internal
and may change with every revision of BProc.

In BProc 3, the goal was to create a message structure which would
allow for a very simple message routers in the daemons and kernel
layers.  Earlier versions of the BProc message routers (the master and
slave daemons and to a lesser extent the char device code in the
kernel) were huge complicated case statements.  Adding a new message
type to the system required updating in 6 or more places even though
only two places really need to know about the message type at all.

To accomplish this goal, we will try to improve the request header so
that the routing pieces can get the information they need without
having to know anything about what the payload of the message is.
(Part of this improvement will be the addition of generic
source/destination information.)

\section{Definitions}
\begin{description}

\item{Real process}\\
In BProc there is a ghost and real process.  The real process is the
one where the computation actually happens.  The ghost is a place
holder in the process table on the front end which represents the
remote process and does some work on its behalf.

\item{Ghost process}\\
This is the place holder process on the front end.

\item{Request}\\
A message to a remote entity which requires a response.  The requester
will usually block in the (D) state while waiting for the response.

\item{Notification}\\
A message which has no response.
\end{description}


\section{Message Assumptions:}

The message handling infrastructure is built with the following assumptions:

\begin{itemize}

\item Ghosts cannot emit messages which require responses.

\item Ghosts only emit messags which are for their corresponding real process.

\item Ghosts can only receive requests from their real process.  That
    is ghosts only talk to their real processes.

\item Real processes cannot be the destination of any messages.  This
    is because they are normal processes.  They are not prepared to
    handle a request from anywhere.  There are exceptions to this
    rule.  In those cases, the message is actually handled by the
    slave or master daemon.

\item Ghosts all exist on the master node.

\item Real processes may exist either on the master node or a slave node.

\item No process has more than one outstanding request at a time.  The
    only exception to this rule is GET\_STATUS but responses to
    GET\_STATUS are not handled via the usual mechanisms.

\end{itemize}

\subsection{Automatic Error Cleanup By Master Daemon}

The master daemon keeps track of outstanding requests so that it can
automatically create an error response in case the slave daemon that
the request was sent to goes down.  In order to automatically generate
an error response the message must be:

\begin{itemize}
\item from a real process or a ghost.  (i.e. not from a node)
\item to a node or another real process.  (i.e. not to a ghost)
\end{itemize}

If the slave that message was sent to goes down, the master will
automatically generate a response with \texttt{-EIO} in the result
field.


\newcommand{\msg}[1]{\texttt{#1}
                     \index{\texttt{#1}}
                     \index{\texttt{BPROC\_#1}}
% Need this but need to figure out how to sanitize text inline
%                     \label{msg:#1}
}

\begin{tabular}{lcccp{1.5in}}
\hline
Message Name & Resp & From & To & Description\\
\hline
\msg{CHILD\_ADD}           & N & R    & R   \\
\msg{CHILD\_DEL}           & N & R    & R   \\
\msg{CONT}                 & N & Rs   & G   & Masq process continue. \\
\msg{EXEC}                 & Y & Rs   & G   & Masq process stopped. \\
\msg{EXIT}                 & N & Rs   & G   & Exit notification. \\
\msg{FWD\_SIG}             & N & G    & Rs  & Get remote process status \\
\msg{GET\_STATUS}
\footnote{GET\_STATUS requests get generated when a process accesses the /proc
   file system on the front end node.  (Ordinary programs like \texttt{ps}
   will generate these.)  Like FWD\_SIG, these are directed to a
   particular PID but are actually handled on the slave daemon where
   the real process for that PID exists.}  & Y & Rm   & Rs/S  \\
\msg{ISORPHANEDPGRP}       & Y & R    & G   \\
\msg{MOVE}                 & Y & R    & S/G \\
\msg{MOVE\_COMPLETE}       & Y & R    & R   \\
\msg{NODE\_CHROOT}         & Y & R    & S   \\
\msg{NODE\_EOF}            & N & M/S  & M/S \\
\msg{NODE\_HALT}           & Y & R    & S   \\
\msg{NODE\_INFO}           & Y & R    & M   \\
\msg{NODE\_PING}           & Y & M/S  & M/S \\
\msg{NODE\_PWROFF}         & Y & R    & S   \\
\msg{NODE\_RECONNECT}      & Y & R    & S   \\
\msg{NODE\_REBOOT}         & Y & R    & S   \\
\msg{NOTIFY}               & Y & M    & S   \\
\msg{PARENT\_EXIT}         & N & R    & Rs  \\
\msg{PGRP\_CHANGE}         & N & R    & R   \\
\msg{PROC\_INFO}           & Y & R    & M   \\
\msg{PTRACE}               & Y & R    & S/G & \tiny{Remote \texttt{ptrace} syscall.}\\
\msg{SET\_CREDS}           & Y & R    & G   \\
\msg{STOP}                 & N & Rs   & G   \\
\msg{SYS\_FORK}            & Y & Rs   & G   \\
\msg{SYS\_KILL}            & Y & Rs   & G   \\
\msg{SYS\_GETPGID}         & Y & Rs   & G   \\
\msg{SYS\_GETSID}          & Y & Rs   & G   \\
\msg{SYS\_SETPGID}         & Y & Rs   & G   \\
\msg{SYS\_SETSID}          & Y & Rs   & G   \\
\msg{WAIT}                 & N & Rs   & G   & \texttt{wait} notify\\
\hline
\end{tabular}

\begin{tabular}{ll}
Dest & Description \\
\hline
G  & Ghost process \\
M  & Master daemon \\
R  & Real process on the master or a slave \\
Rm & Real process on the master \\
Rs & Real process on a slave \\
S  & Slave daemon
\end{tabular}

\section{Detailed message descriptions}

This section describes each of the BProc messages in detail.  Each
message has a description of the message's purpose and list of which
fields in the BProc request structure are relevant for each request
and response.

\begin{description}

\newcommand{\foowidth}{2.5in}

% Macros for formatting the request/response argument blocks
\renewcommand{\msg}[1]{\item{\texttt{#1}}
                       \index{\texttt{#1}}
                       \index{\texttt{BPROC\_#1}}
% Need this but need to figure out how to sanitize text inline
%                     \label{msg:#1}
                       \nopagebreak
}
\newcommand{\pD}{master or slave daemon {}}
\newcommand{\pG}{ghost process {}}
\newcommand{\pM}{master daemon {}}
\newcommand{\pR}{real process on master or slave {}}
\newcommand{\pRm}{real process on master {}}
\newcommand{\pRs}{real process on slave {}}
\newcommand{\pS}{slave daemon {}}

\newcommand{\sa}[1]{See also \texttt{#1}.}
\newcommand{\response}{\multicolumn{3}{l}{Response}\\}
\newcommand{\noresponse}{
  \multicolumn{3}{l}{Response}\\
  \qquad & \multicolumn{2}{l}{\emph{No Response}} \\
}
\newcommand{\norequest}{
  \qquad & \multicolumn{2}{l}{\emph{No Request}} \\
}
\newcommand{\rr}[2]{\qquad & #1 & #2 \\}
\newenvironment{reqresp}{
\begin{tabular}{llp{\foowidth}}
\multicolumn{3}{l}{Request}\\
}{
\end{tabular}
}

\msg{CHILD\_ADD}

  Add a child process to a remote child.  This message updates a
  remote process's non-local child counter.  This message is used when
  \texttt{clone} is called with \texttt{CLONE\_PARENT},
  \texttt{CLONE\_THREAD}, \texttt{CLONE\_PTRACE} or any other flag
  which causes a process other than the \texttt{clone} caller to
  become the new parent process.

  \begin{reqresp}
  \rr{TO  }{\pRs}
  \rr{FROM}{\pRm or \pG}
  \response
  \rr{TO  }{\pRm or \pG}
  \rr{FROM}{\pRs}
  \end{reqresp}

\msg{CONT}

  When a remote process wakes up from a stopped state, it sends this
  message to the ghost process on the front end.  The ghost on the
  front end updates its state to reflect this change.  \sa{STOP}

  \begin{reqresp}
  \rr{TO  }{\pG}
  \rr{FROM}{\pRs}
  \noresponse
  \end{reqresp}

\msg{EXEC}

  \texttt{EXEC} is sent to the ghost process to request that the ghost
  perform an \texttt{execve} system call and transfer the resulting
  memory image back to the slave.  This request closely resembles a
  \texttt{MOVE} call except that the calling process never leaves the
  node that it is on.

  \begin{reqresp}
  \rr{TO             }{\pG}
  \rr{FROM           }{\pRs}
  \rr{bpr\_move\_addr}{IPv4 Address to connect to for
                       \texttt{execve} arguments}
  \rr{bpr\_move\_port}{IPv4 port number to connect to for
                       \texttt{execve} arguments}
  \response
  \rr{TO    }{\pRs}
  \rr{FROM  }{\pG}
  \rr{result}{The result of the \texttt{execve} and subsequent send
              calls.  zero indicates success. }
  \end{reqresp}

\msg{EXIT}

  When a process running on a slave node exits, it sends a
  \texttt{EXIT} message to its ghost on the front end.  This instructs
  the ghost to exit with the same status as the real process.

  If the \texttt{BPROC\_SILENT\_EXIT} bit is set in the exit status,
  the ghost will silently remove itself from the process tree on the
  front end.  The parent will not receive the exit signal and will not
  see the exited process with \texttt{wait}.  This is used internally
  by \texttt{bproc\_rfork} to cleanup failures when called on slave
  nodes.

  \begin{reqresp}
  \rr{TO  }{\pG}
  \rr{FROM}{\pRs}
  \rr{result}{exit status as returned from \texttt{wait}.}
  \rr{bpr\_status\_utime }{Total user time used by the process. }
  \rr{bpr\_status\_stime }{Total system time used by the process. }
  \rr{bpr\_status\_cutime}{Total user time used by children of the process. }
  \rr{bpr\_status\_cstime}{Total system time used by children of the process. }
  \noresponse
  \end{reqresp}

\msg{FWD\_SIG}

  The ghost sends this message to the real process to forward a signal
  it has received.  This signal has already passed permission checks
  and will not be checked further before delivery.

  \begin{reqresp}
  \rr{TO}{\pRs}
  \rr{FROM}{\pG}
  \rr{bpr\_sig\_info}{packed signal information}
  \noresponse
  \end{reqresp}

\msg{GET\_STATUS}

  Processes running on the master send \texttt{GET\_STATUS} messages
  when they access \texttt{/proc} entries for ghost processes.  There
  are two forms of the \texttt{GET\_STATUS} message.  One is a status
  request for a single process, the other is a global status update
  request.

  \textbf{SPECIAL CASE ROUTING:} This message violates all the norms
  for message traffic in the system.  In the case the message is
  routed to a specific remote process, the slave daemon will respond
  with status for that process.  In the case that the message is sent
  to the master daemon, the master daemon copies the message to each
  slave daemon.  The slave daemon will then send a response for
  \emph{every} process that that slave is managing.

  Responses are routed to the master node, not the requesting process.
  The requesting process receives the response by waiting for the
  ghost's status to update.  This allows multiple requesting processes
  to receive status without duplicating message traffic on the
  network.

  \begin{reqresp}
  \rr{TO  }{\pRs or \pM}
  \rr{FROM}{\pM}
  \response
  \rr{TO}{\pM}
  \rr{FROM}{\pRs}
  \rr{bpr\_status\_state }{The process's state (sleeping, running,
                           disk wait, stopped, etc.) }
  \rr{bpr\_status\_utime }{Total user time used by the process. }
  \rr{bpr\_status\_stime }{Total system time used by the process. }
  \rr{bpr\_status\_cutime}{Total user time used by children of the process. }
  \rr{bpr\_status\_cstime}{Total system time used by children of the process. }
  \end{reqresp}

\msg{ISORPHANEDPGRP}

  This message asks the master node if a particular process group is
  orphaned.  The master needs to answer this question since checking
  the entire process tree is required to answer it.  This is relevant
  for signal delivery.

  \begin{reqresp}
  \rr{TO             }{\pM}
  \rr{FROM           }{\pRs}
  \rr{bpr\_pgrp\_pgid}{The process group to query}
  \response
  \rr{TO    }{\pRs}
  \rr{FROM  }{\pM}
  \rr{result}{non-zero is the process group is orphaned, otherwise zero}
  \end{reqresp}

\msg{MOVE}

  This is the migration request message.  It is sent from a real
  process to one of the daemons or the real process's ghost on the
  front end to initiate migration.

  \begin{reqresp}
  \rr{TO                  }{\pR}
  \rr{FROM                }{\pS or \pG}
  \rr{bpr\_move\_oppid    }{original parent process ID.}
  \rr{bpr\_move\_ppid     }{current parent process ID.}
  \rr{bpr\_move\_pgrp     }{process process group ID}
  \rr{bpr\_move\_session  }{process session ID}
  \rr{bpr\_move\_exit\_sig}{process exit signal (i.e. SIGCHLD)}
  \rr{bpr\_move\_addr     }{IPv4 address to connect back to for process data}
  \rr{bpr\_move\_port     }{Port number to connect back to for process data}
  \rr{bpr\_move\_euid     }{Effective user ID of process}
  \rr{bpr\_move\_egid     }{Effective group ID of process}
  \rr{bpr\_move\_groups   }{Supplementary groups}
  \rr{bpr\_move\_children }{Total number of child processes}
  \rr{bpr\_move\_resend   }{Number of times to resend process data.}
  \response
  \rr{TO                  }{\pR}
  \rr{FROM                }{\pS or \pG}
  \rr{result              }{result of move operation.  zero indicates success.}
  \rr{bpr\_move\_r\_addr  }{IPv4 address to connect to for resends.}
  \rr{bpr\_move\_r\_port  }{IPv4 port number to connect to for resends.}
  \rr{bpr\_move\_r\_comm  }{Value of comm after receiving move data.}
  \rr{bpr\_move\_r\_creds }{Process credentials after receiving data.}
  \end{reqresp}

\msg{MOVE\_COMPLETE}

  This message is another strange message.  It is used only in
  response form.  It is used to finish the vector move types and to
  clean up after failed moves in pretty much all cases.

  In a vector move, the receiver will sit around waiting for resend
  requests until a \texttt{MOVE\_COMPLETE} is received telling it to
  go on.

  In failed moves of any kind, if the sender is the node that goes
  away, a \texttt{MOVE\_COMPLETE} is sent to the receiver telling it
  to go on.

  This message is only ever used in the response form for a number of
  reasons.  First of all, ghost is the only process type that comes
  with a request queue.  For real processes, the only way to receive
  messages is to have an outstanding request.  Also, in the vector
  move cases, having an outstanding \texttt{MOVE\_COMPLETE} request
  allows us to use existing error recovery infrastructure to clean up
  after node failures.

  \begin{reqresp}
  \norequest
  \response
  \rr{TO  }{\pR}
  \rr{FROM}{\pR or \pM}
  \rr{message ID}{from corresponding move or PID.}
  \end{reqresp}

\msg{NODE\_CHGRP}

  This message is send to the master daemon to change the group owner
  of a slave node.  The group to be changed to is checked for validity
  before sending the message.  The master daemon compares the euid in
  the request to the uid on the node to see if the change is allowed.

  \begin{reqresp}
  \rr{TO  }{\pM}
  \rr{FROM}{\pR}
  \rr{data.setnode.node}{The node number to modify}
  \rr{data.setnode.euid}{The euid of the process requesting the change.}
  \rr{data.setnode.data}{The group ID to change the group ID to.}
  \response
  \rr{FROM}{A real process}
  \rr{TO }{The master daemon}
  \rr{result}{zero indicates success}
  \end{reqresp}

\msg{NODE\_CHMOD}

\msg{NODE\_CHOWN}

\msg{NODE\_CHROOT}

  This message requests the slave daemon to perform a \texttt{chroot}
  call.  The new root directory will affect all processes migrated to
  the node after the call is completed.

  \begin{reqresp}
  \rr{TO}{\pS}
  \rr{FROM}{\pR}
  \rr{bpr\_string}{path to \texttt{chroot} to.}
  \response
  \rr{FROM}{\pR}
  \rr{TO}{\pS}
  \rr{result}{result of \texttt{chroot}.  zero indicates success}
  \end{reqresp}

\msg{NODE\_EOF}

  This message is sent between daemons before closing a connection to
  indicate a clean shutdown.  None of the fields are used.

\msg{NODE\_FILEREQ\_PFAIL}
\msg{NODE\_FILEREQ\_POK}

\msg{NODE\_HALT}

\msg{NODE\_INFO}

\msg{NODE\_PING}

  This message is send between daemons periodically to verify that the
  connection is still up.  Ping messages from the master daemon
  contain the desired ping timeout.  Slave daemons update their ping
  timeout based on this value.

  \begin{reqresp}
  \rr{TO}{\pD}
  \rr{FROM}{\pD}
  \rr{result}{pings from master only have ping timeout.}
  \response
  \rr{FROM}{\pD}
  \rr{TO}{\pD}
  \end{reqresp}

\msg{NODE\_PWROFF}

  \begin{reqresp}
  \rr{TO}{\pS}
  \rr{FROM}{\pR}
  \end{reqresp}

  This message is sent by a process to a slave daemon to request that
  the slave daemon power off the machine.

\msg{NODE\_RECONNECT}

\msg{NODE\_REBOOT}


\msg{NODE\_STATUS}

\msg{NOTIFY}

  This message is emitted by the master daemon when a machine status
  change occurs.  (i.e. node status change)

  \begin{reqresp}
  \rr{TO}{\pS}
  \rr{FROM}{\pM}
  \noresponse
  \end{reqresp}

\msg{PARENT\_EXIT}

  The master will send this message to a process running on a remote
  machine if its parent process exits.  This notifies the remote
  process that one of the process's process IDs has changed to 1.  The
  message is sent from the process that exited to the real remote
  process which was its child.

  \textbf{SPECIAL CASE ROUTING:} The message originates as a message
  to the master daemon.  The master daemon (which keeps track of
  parent process IDs) determines which remote processes the message
  should be delivered to.  The message is from the real parent process
  which exited.  Even though that process pay have exited on a slave
  node, the message always originates on the master node.

  \begin{reqresp}
  \rr{TO}{\pM / \pRs (child process)}
  \rr{FROM}{\pR (parent process which exited)}
  \noresponse
  \end{reqresp}

\msg{PROC\_INFO}


\msg{PTRACE}

\msg{SET\_CREDS}

  When a process running a slave node performs an operation which
  modifies the process credentials, this message is sent to the ghost
  process to update the credentials on the ghost.  These operations
  include the \texttt{set*id} system calls and \texttt{execve}.

  \begin{reqresp}
  \rr{TO}{\pG}
  \rr{FROM}{\pRs}
  \rr{bpr\_proc\_creds}{packed process credentials}
  \noresponse
  \end{reqresp}

\msg{STOP}

  This message is send to the ghost process when the real process
  enters the stopped state (T).  The ghost updates its status to
  reflect the new status.

  \begin{reqresp}
  \rr{TO}{\pG}
  \rr{FROM}{\pRs}
  \noresponse
  \end{reqresp}

\msg{SYS\_FORK}

   When a process running on a slave node \texttt{fork}s, this message
   is sent to the ghost process to also perform the same fork.  This
   keeps both process trees synchronized and also allocates a process
   ID for the child process on the remote node.

   This message also covers the \texttt{clone} system call.

   \begin{reqresp}
   \rr{TO}{\pG}
   \rr{FROM}{\pRs}
   \rr{bpr\_rsyscall\_arg[0]}{clone flags}
   \response
   \rr{TO}{\pRs}
   \rr{FROM}{\pG}
   \rr{result}{new process ID of child or error if negative}
   \end{reqresp}

\msg{SYS\_KILL}
\msg{SYS\_GETPGID}
\msg{SYS\_GETSID}
\msg{SYS\_SETPGID}
\msg{SYS\_SETSID}
\msg{WAIT}

   When a process running on a slave succesfully performs a
   \texttt{wait} on a zombie process, the slave sends a \texttt{WAIT}
   message to the master to update the process tree there as well.
   The same \texttt{wait} call is performed by the ghost process.

   \begin{reqresp}
   \rr{TO                   }{\pG}
   \rr{FROM                 }{\pRs}
   \rr{bpr\_rsyscall\_arg[0]}{process ID to wait on}
   \rr{bpr\_rsyscall\_arg[1]}{\texttt{wait} flags with
                               \texttt{WNOHANG} masked off}
   \noresponse
   \end{reqresp}

\end{description}


% Message Name            Resp?   From            To              Short descriptio
% n
% --------------------------------------------------------------------------------
% ----------------------------------------------------
% BPROC_CONT              N       Real            Ghost           Process continue d
% BPROC_EXEC              N       Real            Ghost           Notification of task_struct->comm change.
% BPROC_EXIT              N       Real            Ghost           Process exit with status.
% BPROC_FWD_SIG           N       Ghost           Real*           Forward a signal
%  received by the ghost to the remote process.
%                         N       Real***         Ghost           TO BE IMPLEMENTE
% D....
% BPROC_GET_STATUS        Y       Real(M)**       Slave Daemon    Get status of a 
% remote process.
% BPROC_GETPGID           Y       Real            Ghost
% BPROC_GETPPID           Y       Real            Ghost
% BPROC_GETSID            Y       Real            Ghost
% BPROC_ISORPHANEDPGRP    Y       Real            Master
% BPROC_MOVE              Y       Real            Node            Move a process t
% o a node.
% BPROC_SET_CREDS         N       Real            Ghost           Process did setu
% id, getgid, etc. /* Not implemented */
% BPROC_SETPGID           Y       Real            Ghost
% BPROC_SETSID            Y       Real            Ghost
% BPROC_STOP              N       Real            Ghost           Process stopped
% BPROC_SYS_FORK          Y       Real            Ghost           Have the ghost f
% ork to allocate a new process ID.
% BPROC_SYS_KILL          Y       Real            Ghost           Perform a remote
%  signal delivery.
% BPROC_SYS_WAIT          Y       Real            Ghost           Perform a wait()

% BPROC_PTRACE            Y       Real(M)         Real            Remote ptrace sy
% scalls.
% BPROC_PTRACE_DETACH     N       Real(M)         Real            A silent process
%  detach.  (used by do_exit)

%\printindex
%\end{document}
