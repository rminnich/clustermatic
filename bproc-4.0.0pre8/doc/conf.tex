%% $Id: conf.tex,v 1.3 2003/10/22 23:29:26 mkdist Exp $
\chapter{Configuration Reference}

The BProc master daemon and beoserv read their configuration
information from \texttt{/etc/clustermatic/config}.  This file is a plain
text file.  All configuration directives are a single line take the
form of:\\
\textbf{tag} \textit{arguments ...}

Comments are shell script style, beginning with \# and extending to the
end of the line.  Since several programs and scripts use this
configuration file, the BProc master daemon will only pay attention to
tags it recognizes and ignore all others.  An example configuration
file follows.

\section{Configuration Directives}

\begin{description}

\item{\texttt{bprocport} \emph{portnumber}}
\label{conf:bprocport}
\index{\texttt{bprocport}}

The \texttt{bprocport} command tells the BProc master daemon which TCP
port it should listen on for connections from slave machines.  Only
one \texttt{bprocport} command is allowed.  The master daemon will
listen on this port on all interfaces.  (See \texttt{interface} on
page \pageref{conf:interface})

\item{\texttt{interface} \emph{interfacename} \emph{IPaddress} \emph{netmask}}
\label{conf:interface}
\index{\texttt{interface}}

The \texttt{interface} command tells the BProc master daemon
and the beoserv daemon to listen for connections on a particular
interface.  If multiple interface lines appear in the file, the
daemons will listen on multiple interfaces.


\item{\texttt{ip} [\emph{nodenumber}] \emph{IPaddress ... }}
\label{conf:ip}
\index{\texttt{ip}}

The \texttt{ip} assigns one or more IP addresses to a single node.  If
the \emph{nodenumber} argument is given, the IP address will be
assigned to that node.  If it is omitted, it will be assigned to the
node following whatever node was assigned last.  If no nodes have been
assigned, assignment will start with node zero.  IP addresses must be
in dotted notation. (ie. 10.0.0.1) No hostnames are allowed.

\item{\texttt{iprange} [\emph{nodenumber}] \emph{IPaddress1} \emph{IPaddress2}}
\label{conf:iprange}
\index{\texttt{iprange}}

The \emph{iprange} assigns addresses \emph{ip\_address1} up to and
including \emph{ip\_address2} to nodes.  If the \emph{node\_number}
argument is given, the first address will be assigned to that node,
the next address to the next node and so on.  If the
\emph{node\_number} argument is omitted, the addresses will be assigned
starting with the node following whatever node was assigned last.  If
no nodes have been assigned, assignment will start with node zero.

\item{\texttt{logfacility} \emph{facility}}
\label{conf:logfacility}
\index{\texttt{logfacility}}

The \texttt{logfacility} command specificies which log facility the
BProc master daemon should use.  Some valid log facility names are
\texttt{daemon}, \texttt{syslog} and \texttt{local0}.  See the syslog
documentation for the complete set.  By default \texttt{daemon} will
be used.

\item{\texttt{mcastbcast} \emph{interface}}
\label{conf:mcastbcast}
\index{\texttt{mcastbcast}}

The \texttt{mcastbcast} command tells beoserv to use broadcast instead
of multicast when transmitting files on \emph{interface}.  This is
provided since some network equipment (interface cards, switches,
etc.) seem to have trouble with large volumes of multicast traffic.

\item{\texttt{mcastport} \emph{portnumber}}
\label{conf:mcastport}
\index{\texttt{mcastport}}

The \texttt{mcastport} command tells the beoserv daemon which port it
should listen on for file requests from the nodes.  Only one
\texttt{mcastport} command is allowed.

\item{\texttt{mcastthrottle} \emph{interface} \emph{mbits}}
\label{conf:mcastthrottle}
\index{\texttt{mcastthrottle}}

The \texttt{mcastthrottle} command controls the speed at which data is
transmitted on a particular interface.  The speed is given in megabits
per second.  Use this if your front end's network interface is faster
than your slaves' interfaces and you need to slow down file sends.

\item{\texttt{node} [\emph{nodenumber}] \emph{MACaddrs ...}}
\label{conf:node}
\index{\texttt{node}}

The \texttt{node} command assigns MAC addresses to node numbers.

\item{\texttt{nodes} \emph{number}}
\label{conf:nodes}
\index{\texttt{nodes}}

The \texttt{nodes} command specificies how many nodes will be in the
system.  The nodes will be numbered from 0 through \emph{number}-1.
Addresses are assigned to these nodes with the \texttt{ip} and
\texttt{iprange} commands.  This numbering does not include the
master.

%----------------------------
% nodeup* for the beoboot node_setup thing
\item{\texttt{nodeupmaxclients}} \emph{number}
\label{conf:nodeupmaxclients}
\index{\texttt{nodeupmaxclients}}

Beoboot does node setup in groups.  The \texttt{nodeupmaxclients}
determines the largest group of nodes that will be setup at once.
Normally this number should be as big as possible to take advantage of
the mass process creation facilities in BProc.  There is, however, a
limit imposed by the maximum number of open a files a process can
have.  Don't change this unless you know what you're doing.

\item{\texttt{nodeupmaxworkers}} \emph{number}
\label{conf:nodeupmaxworkers}
\index{\texttt{nodeupmaxworkers}}

The \texttt{nodeupmaxworkers} directive controls the maximum number of
threads which can be busy on the front end at any given time.  This
essentially controls the maximum load that the node setup scripts can
place on the front end.

\item{\texttt{nodeupworker}} \emph{/path/to/node\_up}
\label{conf:nodeupworker}
\index{\texttt{nodeupworker}}

The \texttt{nodeupworker} directive tells beoboot the path to the node
setup program.
%----------------------------

\item{\texttt{pingtimeout} \emph{seconds}}
\label{conf:pingtimeout}
\index{\texttt{pingtimeout}}

The \texttt{pingtimeout} command controls how frequently the BProc
master and slave daemons will ping one another to make sure they're
still alive.  This value is propaged to the slave machines when they
connect to the master.

\end{description}

%\end{document}
