% $Id: procacct.tex,v 1.2 2002/07/23 20:38:48 mkdist Exp $
\chapter{Process Accounting}

BProc adds some process accounting infrastructure on the slave nodes
of the cluster.  Process accounting on the master is done using the
normal UNIX process infrastructure.  The existing UNIX infrastructure
is used to represent parent/child relationships on the slave node.
The slave node is more complicated since any process running on the
slave node may have parent and/or child processes which do not exist
on the same node.  These cased need to be handled transparently.

\section{Slave node process trees and \texttt{nlchild}}
\index{\texttt{nlchild}}
When a process running on a slave node calls \texttt{wait}, the child
it is waiting on may or may not exist on the same node.  If the child
does exist on the slave node, BProc should handle the wait locally to
minimize the delay and involvement of the master node.  In order to
know whether or not a \texttt{wait} can be handled locally, the slave
node needs to know if a process has any children which exist on other
nodes.  BProc introduces a child counting mechanism on the slave node
to answer this question.

In general the process accounting is setup with the following
assumptions\footnote{These assumptions are violated by some of the
thread group infrastructure which appeared in Linux 2.4.x.  BProc's
behavior with programs that use those features is currently
undefined.}:
\begin{itemize}
\item{No process (except init) will see an increase in its number of
child processes without calling \texttt{fork} or \texttt{ptrace}}
\item{No process will lose children without calling \texttt{wait}}\\
\end{itemize}

The PID masquerading code adds a counter on the slave node
(\texttt{nlchild}) which counts the number of non-local children.  A
non-local child is a child process which exists on another node and
therefore isn't visible on the slave node.  When a process moves to a
slave node, its total number of children is sent along with the move
request.  When the process is inserted into the process tree on the
new node, \texttt{nlchild} is initialized with the total number of
children.  Then the parent process IDs of the other processes on the
slave are inspected.  If any of those procesess are children of the
process moving in, the parent process pointers and \texttt{nlchild}
are updated accordingly.  The two assumptions above guarantee that the
number of children will not change during a move.

The \texttt{ptrace} attach would appear to violate the second
assumption above since it appears to steal a child from another
process.  It is true that for the purposes of \texttt{wait} the child
becomes the child of the tracer.  BProc doesn't treat
\texttt{ptrace} attach as a child removal because the child will
always be given back when the process doing the trace detaches.

% POSSIBLE (Linux) BUG:  What about wait()s initiated after a child
% has been attached?  It seems we should be able to get an ECHLD here
% when we shouldn't.

\section{Parent processes, \texttt{child\_reaper} and \texttt{ptrace}}
\index{\texttt{child\_reaper}}
\index{\texttt{ptrace}}
In most cases when a process migrates to a slave node, its parent
process will not be present on the same node.  In this case, the BProc
slave daemon will be the parent process in the slave's process tree.
The \texttt{getppid} system call will still return the proper process
ID.  \emph{The slave daemon is the parent process for any process
whose real parent is not present.}  This allows the slave daemon to
perform \texttt{ptrace} requests for remote processes.  If a process's
parent process exits, it is reparented to the slave daemon, not the
usual \texttt{child\_reaper} on the slave node.

\section{Updating parent process IDs}

When a process moves to a remote node, it takes its own process ID and
its parent process ID with it.  The parent process may change during
the life of the process.  For example a process running on a slave
node may have a parent process which is running on the front end.  If
the parent process exits, the remote process should have its parent
process ID changed.

\index{\texttt{PARENT\_EXIT}}
The master sends out a single \texttt{PARENT\_EXIT} message when a
process with ghost children exits on the master.  The master's kernel
doesn't emit a message for every child because the number of children
could potentially be \emph{very} large.  The master daemon then
determines which nodes the message should be forwarded to.  The slave
daemons then update the parent process IDs in their local process
trees.

% explanation of the negative PID hack.
